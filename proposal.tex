\documentclass{article}

  % if you need to pass options to natbib, use, e.g.:
  % \PassOptionsToPackage{numbers, compress}{natbib}
  % before loading nips_2018
  
  % ready for submission
  %\PassOptionsToPackage{numbers, compress}{natbib}
  \usepackage[preprint]{nips_2018}
  
  % to compile a preprint version, e.g., for submission to arXiv, add
  % add the [preprint] option:
  % \usepackage[preprint]{nips_2018}
  
  % to compile a camera-ready version, add the [final] option, e.g.:
  % \usepackage[final]{nips_2018}

  % to avoid loading the natbib package, add option nonatbib:
  % \usepackage[nonatbib]{nips_2018}
  
  \usepackage[utf8]{inputenc} % allow utf-8 input
  \usepackage[T1]{fontenc}    % use 8-bit T1 fonts
  \usepackage{hyperref}       % hyperlinks
  \usepackage{url}            % simple URL typesetting
  \usepackage{booktabs}       % professional-quality tables
  \usepackage{amsfonts}       % blackboard math symbols
  \usepackage{nicefrac}       % compact symbols for 1/2, etc.
  \usepackage{microtype}      % microtypography
  %\usepackage{biblatex}
  %\addbibresource{proposal.bib}

  \title{Online Hate Speech Detection on Twitter}
  
  % The \author macro works with any number of authors. There are two
  % commands used to separate the names and addresses of multiple
  % authors: \And and \AND.
  %
  % Using \And between authors leaves it to LaTeX to determine where to
  % break the lines. Using \AND forces a line break at that point. So,
  % if LaTeX puts 3 of 4 authors names on the first line, and the last
  % on the second line, try using \AND instead of \And before the third
  % author name.
  
  \author{
    Jainul N.~Vaghasia\\
    Department of Computer Science\\
    University of Washington\\
    Seattle, WA 98105 \\
    \texttt{jnv3@cs.washington.edu}
    %% examples of more authors
    %% \And
    %% Coauthor \\
    %% Affiliation \\
    %% Address \\
    %% \texttt{email} \\
    %% \AND
    %% Coauthor \\
    %% Affiliation \\
    %% Address \\
    %% \texttt{email} \\
    %% \And
    %% Coauthor \\
    %% Affiliation \\
    %% Address \\
    %% \texttt{email} \\
    %% \And
    %% Coauthor \\
    %% Affiliation \\
    %% Address \\
    %% \texttt{email} \\
  }
  
  \begin{document}
  % \nipsfinalcopy is no longer used
  
  \maketitle
  \section{Proposal}
  \subsection{Datasets}
  Sexism/Racism dataset (\cite{waseem-hovy:2016:N16-2}), Hate dataset (\cite{hateoffensive}), expecting use of synthetic data to explore additional behavior.
  \subsection{Idea}
  Determining whether a tweet is offensive or not is an online binary classification problem that is
  inherently challenged by class imbalance and concept drift---it is expected that offensive tweets appear
  less often than not, and that the underlying distribution of such tweets varies based on socio-political events (\cite{Golbeck:2017:LLC:3091478.3091509}).
  In this project, we explore an online approach based on selective resampling of a subset of past training data,
  and apply it to a neural network classifier based on word embeddings for predicting the class, offensive or not, of a tweet.
  \cite{Malialis2018QueueBasedRF}'s novel, online algorithm, albeit simple, is highly intuitive and produces high quality results on synthetic data, whereas
  \cite{kshirsagar2018predictive}'s work uses relatively fewer parameters than its counterparts on offline model while achieving high F1 scores.
  
  We stress-test the union of both ideas
  and explore whether it works equally well while addressing the aforementioned practical problems one might encounter with the given classification problem.
  If this approach works, what does each idea bring to the approach?
  Do we experience any counteraction on working of one algorithm as a result of the other one and vice-versa?
  If the approach fails, what makes this problem uniquely difficult to solve: online vs. offline, magnitude of imbalance, type of drift? We wish
  to explore these potential questions in this project.
  \subsection{Software}
  We expect to implement all the required algorithms and model online data streams using basic Python libraries like Numpy, Scipy, etc.

  \subsection{Expected Progress by Milestone}
  By the Milestone, we expect to produce experimental evidence on effectiveness of the approach on online class imbalance.

  \bibliographystyle{plainnat}
  \bibliography{proposal}

  \end{document}